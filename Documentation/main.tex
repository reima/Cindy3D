\documentclass{scrartcl}
\usepackage{amsmath,amsfonts,amssymb}
\usepackage{mathpazo}
\usepackage[utf8]{inputenc}

\title{Cindy3D Project Documentation}
\author{Matthias Reitinger \and Jan Sommer}

\begin{document}
\maketitle

\newpage

\tableofcontents

\newpage

\section{Project overview}

\subsection{Cinderella \& CindyScript}

\emph{Cinderella} is a dynamic geometry software created by Ulrich Kortenkamp and Jürgen Richter-Gebert. One of its key features is the embedded functional scripting language \emph{CindyScript}. It enables users of \emph{Cinderella} to interact with geometric constructions in a programmatic way. Among others, \emph{CindyScript} provides a rich set of methods for drawing two-dimensional geometry. However the display of three-dimensional objects is cumbersome, as one would have to fall back on the 2D drawing methods and implement algorithms like perspective projection and hidden surface removal by hand.

The recent release of \emph{Cinderella 2.6} provides a new plugin interface. This allows programmers to write \emph{Java} libraries which expose \emph{CindyScript} methods that can in turn be called from inside \emph{Cinderella} constructions.

\subsection{Cindy3D}

In this context the \emph{Cindy3D} project was born. The need for 3D drawing functionality in \emph{Cinderella} was in fact one of the main reasons to create the new plugin interface. The first version of \emph{Cindy3D} was developed when the plugin interface was still in a preliminary stage.
% What we wanted to achieve

\subsection{Results}
% What we accomplished

\subsection{Future directions}

There are some features we would have wanted in \emph{Cindy3D}, yet didn't make it into this release for various reasons (like limited time or constraints in \emph{Cinderella} at the time of development). We hope that these features can be implemented in further versions of \emph{Cindy3D}.

\section{User documentation}

\subsection{Installation}

\begin{itemize}
\item Minimum system requirements (OpenGL version...)
\item Install guide (with screenshots?)
\end{itemize}

\subsection{User guide}

\begin{itemize}
\item Understanding of \emph{CindyScript} is assumed
\item Scene management (begin3d, end3d)
\item Settings (renderhints...)
\item Coordinate system
\item Primitive types
\item Appearance (attributes, stack)
\item Examples
\item Tutorial(s)?
\end{itemize}

\subsection{Command reference}

Link to HTML documentation or inline.

\section{Developer documentation}

\subsection{Technical overview}

\begin{itemize}
\item Java 6
\item Used libraries: JOGL, apache-commons-math
\item JAR packaging
\end{itemize}

\subsection{Design overview}

Insert fancy diagram here.
% Translation of calls from Cinderella to a library agnostic format (only types provided by Java or derivatives thereof)

\subsection{Primitive rendering}

\begin{itemize}
\item Explain raytraced rendering with proxy geometry
\item Explain LOD
\end{itemize}

\subsection{JavaDoc}

Link to generated JavaDoc.

\end{document}
