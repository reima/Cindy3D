\documentclass[10pt,a4paper]{scrartcl}
\usepackage[latin1]{inputenc}
\usepackage{amsmath}
\usepackage{amsfonts}
\usepackage{amssymb}
\usepackage{graphicx}
\usepackage{enumerate}
\usepackage{url}
\usepackage[nohead,nofoot,lmargin=2.5cm,rmargin=2.5cm,tmargin=3cm,bmargin=3cm,headsep=0.5cm,headheight=12.4pt, foot=0.75cm]{geometry}

\linespread{1.3}
\parindent 0cm
%\parskip 2.5ex plus0.5ex minus0.5ex
\parskip 2.5ex
\setlength{\itemsep}{100pt}

\author{MR & JS}
\title{W�rstchenfabrik}
\begin{document}
\includegraphics[height=1.5cm]{tum-info-logo}
\parbox[b]{11.7cm}{
\begin{Large}
\textsc{\centering Fakult�t f�r Informatik\vspace{-0.5cm}\begin{center}der Technischen Universit�t M�nchen\end{center}}
\end{Large}}
\includegraphics[height=1.5cm]{tum-logo}
\begin{center}
\textsf{\begin{large}Interdisziplin�res Projekt im Anwendungsfach Mathematik\end{large}\\[1ex]
\begin{Large}
\textsf{\textbf{Anbindung von dynamischer Geometriesoftware an jReality}}\\[1ex]
\textsf{\textbf{Interfacing dynamic geometry software and jReality}}
\end{Large}
}
\vspace{0.5cm}
\end{center}
\hrule
\section{Motivation und Inhalt}
Interaktive Geometriesoftware besitzt vielf�ltige Einsatzgebiete. Dazu z�hlen unter anderem die interaktive Exploration von mathematischen und physikalischen Beziehungen durch den Benutzer und die Visualisierung von geometrischen Zusammenh�ngen.

Das von J�rgen Richter-Gebert und Ullrich Kortenkamp konzipierte, geschriebene und entwickelte Cinderella 2 stellt eine solche Geometriesoftware dar. Sie wird nicht nur in der Lehre an Universit�ten eingesetzt, sondern findet auch im Schulunterricht Anwendung. Eine der St�rken von Cinderella ist die integrierte funktionale Programmiersprache CindyScript, �ber die sich das Verhalten von Cinderella erweitern und anpassen l�sst. Unter anderem ist es mit CindyScript m�glich, programmgesteuert zweidimensionale Zeichnungen zu erzeugen. Die Darstellung dreidimensionaler Zusammenh�nge l�sst sich allerdings aufgrund der fehlenden Standardschnittstelle zur 3D-Unterst�tzung bisher nur mit erheblichem Aufwand und in beschr�nktem Ma�e realisieren.

Ziel dieses interdisziplin�ren Projektes ist es, CindyScript um die M�glichkeit zur Darstellung dreidimensionaler Objekte zu erweitern. Dazu soll ausschlie�lich die in Cinderella enthaltene Plugin-Schnittstelle genutzt werden. Mit dieser k�nnen in der Programmiersprache Java geschriebene Methoden CindyScript zur Verf�gung gestellt werden, die dann transparent f�r Cinderella das eigentliche Rendering der dreidimensionalen Objekte �bernehmen sollen.

Dabei soll auf jReality zur�ckgegriffen werden. Diese auf mathematische und wissenschaftliche Inhalte spezialisierte 3D-Visualisierungsbibliothek stellt ein umfangreiches API zur Verf�gung. Eine der Herausforderungen des Projektes besteht darin, die reichhaltige Funktionalit�t von jReality auf einen reduzierten Satz von Methoden abzubilden, welche dann �ber die Plugin-Schnittstelle von Cinderella aus CindyScript heraus angesprochen werden k�nnen. Hierbei sollen keine jReality-spezifischen Eigenschaften mit einflie�en. Auf diese Weise soll garantiert werden, dass jReality sich sp�ter auch durch eine andere Bibliothek austauschen lie�e.

Dar�ber hinaus soll im Rahmen des Projektes auch die Praxis-Tauglichkeit der Plugin-Schnittstelle von Cinderella ausgelotet werden. Dadurch soll gegebenenfalls ein indirekter Beitrag zur sinnvollen Erweiterung dieses sich noch in der Beta-Phase befindlichen Features geleistet werden.

\parskip 0.5ex
\section{Anforderungen}
\begin{enumerate}[$\triangleright$]
\item \textit{Entwurf einer benutzerfreundlichen Schnittstelle zur Verwendung einer 3D-Visualisierungsbibliothek}
\begin{enumerate}[-]
\item Festlegung eines sinnvollen Satzes von CindyScript-Funktionen die dem Benutzer zur Verf�gung gestellt werden
\item Dabei Anlehnung an die bereits existierenden CindyScript-Funktionen zur 2D-Darstellung (Principle of Least Surprise)
\item Festlegung sinnvoller Standardparameter, die bei Bedarf angepasst werden k�nnen
\item Erstellen einer vollst�ndigen Benutzerdokumentation
\end{enumerate}
\vspace{0.5cm}
\end{enumerate}
\begin{enumerate}[$\triangleright$]
\item \textit{jReality als prim�re Implementierung der oben genannten Schnittstelle}
\begin{enumerate}[-]
\item Unterst�tzung von Grundobjekten:\\Punkte, Strecken, Halbgeraden, Geraden, Polygone, Kreise, Sph�ren und Quadmeshes
\item Freie Positionierung von Objekten im dreidimensionalen Raum
\item Modifikation der Materialparameter der Objekte
\item Darstellung von 3D-Plots von Funktionen
\item Anpassung der Szenenparameter: Kamera, Licht und Hintergrund
\item Erstellen einer vollst�ndigen Entwicklerdokumentation
\end{enumerate}
\end{enumerate}
\section{Weiterf�hrende Informationen}
\begin{enumerate}
\item Cinderella\\\url{http://www.cinderella.de}
\item CindyScript\\\url{http://doc.cinderella.de/tiki-index.php?page=CindyScript}
\item jReality\\\url{http://www3.math.tu-berlin.de/jreality}
\end{enumerate}
\end{document}